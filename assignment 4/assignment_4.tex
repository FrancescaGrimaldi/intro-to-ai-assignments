\documentclass{article}
\usepackage[utf8]{inputenc}
\usepackage{titling}

% Page size and margins
\usepackage[a4paper,top=2cm,bottom=2cm,left=3cm,right=3cm,marginparwidth=1.75cm]{geometry}

% Language setting
\usepackage[english]{babel}

% Useful packages
\usepackage{amsmath}
\usepackage[colorlinks=true, allcolors=blue]{hyperref}
\usepackage{minted}

% Header
\usepackage{fancyhdr}
\pagestyle{fancy}
\fancyhead{} % empties all fields
\fancyhead[R]{Filippini, Grimaldi}
\fancyhead[L]{TDT4136 - Assignment 4}

% Images
\usepackage{graphicx}
\graphicspath{ {images/} }
\usepackage{float}

% Tables
\usepackage{tabu}
\usepackage{caption} 
\captionsetup[table]{skip=2pt}
\usepackage[table]{xcolor}
\usepackage{array}

% Lists
\usepackage{listings}
\usepackage{enumitem}
\setlist{topsep=2pt, itemsep=2pt, partopsep=2pt, parsep=2pt}

% renew/new command
\newcommand{\subtitle}[1]{%
  \posttitle{%
    \par\end{center}
    \begin{center}\large#1\end{center}
    \vskip0.5em}
}

% Title and info
\title{%
    \huge Assignment 4}

\subtitle{%
    TDT4136 - Introduction to Artificial Intelligence \\
    Fall 2023
    }

\author{%
  Elena Filippini\\
  Francesca Grimaldi
}

\date{}


\begin{document}

% TITLE
\maketitle


% INDEX
% \tableofcontents

% INTRODUCTION
\section{Introduction}
In this assignment we implemented the Minimax algorithm following the pseudocode in the AIMA book \cite{ai} in the Pac-Man Projects, developed at UC Berkeley \cite{berkeley}.

More specifically, we completed the classes \texttt{MinimaxAgent} and \texttt{AlphaBetaAgent} with \texttt{maxValue} and \texttt{minValue} functions. These are invoked recursively by \texttt{minimaxSearch} and \texttt{alphabetaSearch} respectively, based on the index of the current agent (which is \texttt{0} if the player is Pacman and a number \texttt{>=1} if it is the turn of one of the ghosts). The search stops when one of the terminal conditions is met: the tree was expanded to the maximum depth, or Pacman won or lost the game.

The algorithm in the \texttt{AlphaBetaAgent} differs from that of \texttt{MinimaxAgent} for the two bounds introduced: $\alpha$ and $\beta$.
$\alpha$ represents the value of the best choice we have found so far at any choice point along the path for MAX, while $\beta$ is the value of the best choice found so far for MIN \cite{ai}.


The code downloaded from \cite{berkeley} includes the program \texttt{autograder.py}, used to run various tests in order to evaluate our solutions. 

The program's outputs are presented in sections \ref{min} and \ref{alfabeta}.

% MINIMAX
\section{Minimax}\label{min}
\begin{figure}[H]
    \centering
    \includegraphics[width=1\textwidth]{img/q2-a.png}
    \caption{Question 2 output (a)}
    \label{fig:q2-a}
\end{figure}

\begin{figure}[H]
    \centering
    \includegraphics[width=1\textwidth]{img/q2-b.png}
    \caption{Question 2 output (b)}
    \label{fig:q2-b}
\end{figure}


% ALPHA-BETA PRUNING
\section{Alpha-Beta Pruning}\label{alfabeta}
\begin{figure}[H]
    \centering
    \includegraphics[width=1\textwidth]{img/q3-a.png}
    \caption{Question 3 output (a)}
    \label{fig:q3-a}
\end{figure}

\begin{figure}[H]
    \centering
    \includegraphics[width=1\textwidth]{img/q3-b.png}
    \caption{Question 3 output (b)}
    \label{fig:q3-b}
\end{figure}

\bibliographystyle{plain}
\bibliography{sample}

\end{document}